\documentclass[]{article}
\usepackage{lmodern}
\usepackage{amssymb,amsmath}
\usepackage{ifxetex,ifluatex}
\usepackage{fixltx2e} % provides \textsubscript
\ifnum 0\ifxetex 1\fi\ifluatex 1\fi=0 % if pdftex
  \usepackage[T1]{fontenc}
  \usepackage[utf8]{inputenc}
\else % if luatex or xelatex
  \ifxetex
    \usepackage{mathspec}
  \else
    \usepackage{fontspec}
  \fi
  \defaultfontfeatures{Ligatures=TeX,Scale=MatchLowercase}
\fi
% use upquote if available, for straight quotes in verbatim environments
\IfFileExists{upquote.sty}{\usepackage{upquote}}{}
% use microtype if available
\IfFileExists{microtype.sty}{%
\usepackage{microtype}
\UseMicrotypeSet[protrusion]{basicmath} % disable protrusion for tt fonts
}{}
\usepackage[margin=1in]{geometry}
\usepackage{hyperref}
\hypersetup{unicode=true,
            pdftitle={Analyse en Composantes Principales},
            pdfauthor={Jeu de données donb},
            pdfborder={0 0 0},
            breaklinks=true}
\urlstyle{same}  % don't use monospace font for urls
\usepackage{graphicx,grffile}
\makeatletter
\def\maxwidth{\ifdim\Gin@nat@width>\linewidth\linewidth\else\Gin@nat@width\fi}
\def\maxheight{\ifdim\Gin@nat@height>\textheight\textheight\else\Gin@nat@height\fi}
\makeatother
% Scale images if necessary, so that they will not overflow the page
% margins by default, and it is still possible to overwrite the defaults
% using explicit options in \includegraphics[width, height, ...]{}
\setkeys{Gin}{width=\maxwidth,height=\maxheight,keepaspectratio}
\IfFileExists{parskip.sty}{%
\usepackage{parskip}
}{% else
\setlength{\parindent}{0pt}
\setlength{\parskip}{6pt plus 2pt minus 1pt}
}
\setlength{\emergencystretch}{3em}  % prevent overfull lines
\providecommand{\tightlist}{%
  \setlength{\itemsep}{0pt}\setlength{\parskip}{0pt}}
\setcounter{secnumdepth}{0}
% Redefines (sub)paragraphs to behave more like sections
\ifx\paragraph\undefined\else
\let\oldparagraph\paragraph
\renewcommand{\paragraph}[1]{\oldparagraph{#1}\mbox{}}
\fi
\ifx\subparagraph\undefined\else
\let\oldsubparagraph\subparagraph
\renewcommand{\subparagraph}[1]{\oldsubparagraph{#1}\mbox{}}
\fi

%%% Use protect on footnotes to avoid problems with footnotes in titles
\let\rmarkdownfootnote\footnote%
\def\footnote{\protect\rmarkdownfootnote}

%%% Change title format to be more compact
\usepackage{titling}

% Create subtitle command for use in maketitle
\newcommand{\subtitle}[1]{
  \posttitle{
    \begin{center}\large#1\end{center}
    }
}

\setlength{\droptitle}{-2em}

  \title{Analyse en Composantes Principales}
    \pretitle{\vspace{\droptitle}\centering\huge}
  \posttitle{\par}
    \author{Jeu de données donb}
    \preauthor{\centering\large\emph}
  \postauthor{\par}
    \date{}
    \predate{}\postdate{}
  

\begin{document}
\maketitle

Ce jeu de données contient 5581 individus et 46 variables.

\begin{center}\rule{0.5\linewidth}{\linethickness}\end{center}

\subsubsection{1. Observation d'individus
extrêmes}\label{observation-dindividus-extremes}

L'analyse des graphes ne révèle aucun individu singulier.

\begin{center}\rule{0.5\linewidth}{\linethickness}\end{center}

\subsubsection{2. Distribution de
l'inertie}\label{distribution-de-linertie}

L'inertie des axes factoriels indique d'une part si les variables sont
structurées et suggère d'autre part le nombre judicieux de composantes
principales à étudier.

Les 2 premiers axes de l' ACP expriment \textbf{47.4\%} de l'inertie
totale du jeu de données ; cela signifie que 47.4\% de la variabilité
totale du nuage des individus (ou des variables) est représentée dans ce
plan. C'est un pourcentage relativement moyen, et le premier plan
représente donc seulement une part de la variabilité contenue dans
l'ensemble du jeu de données actif. Cette valeur est nettement
supérieure à la valeur référence de \textbf{5.13\%}, la variabilité
expliquée par ce plan est donc hautement significative (cette intertie
de référence est le quantile 0.95 de la distribution des pourcentages
d'inertie obtenus en simulant 154 jeux de données aléatoires de
dimensions comparables sur la base d'une distribution normale).

Du fait de ces observations, il serait alors probablement nécessaire de
considérer également les dimensions supérieures ou égales à la troisième
dans l'analyse.

\begin{center}\includegraphics{PCA_files/figure-latex/unnamed-chunk-2-1} \end{center}

\textbf{Figure 2 - Decomposition of the total inertia on the components
of the ACP}

Une estimation du nombre pertinent d'axes à interpréter suggère de
restreindre l'analyse à la description des 10 premiers axes. Ces
composantes révèlent un taux d'inertie supérieur à celle du quantile
0.95 de distributions aléatoires (77.59\% contre 24.59\%). Cette
observation suggère que seuls ces axes sont porteurs d'une véritable
information. En conséquence, la description de l'analyse sera restreinte
à ces seuls axes.

\begin{center}\rule{0.5\linewidth}{\linethickness}\end{center}

\subsubsection{3. Description du plan 1:2}\label{description-du-plan-12}

\begin{center}\includegraphics{PCA_files/figure-latex/unnamed-chunk-3-1} \end{center}

\textbf{Figure 3.1 - Graphe des individus (ACP)} \emph{Les individus
libellés sont ceux ayant la plus grande contribution à la construction
du plan.}

\begin{center}\includegraphics{PCA_files/figure-latex/unnamed-chunk-4-1} \end{center}

\textbf{Figure 3.2 - Graphe des variables (ACP)} \emph{Les variables
libellées sont celles les mieux représentées sur le plan.}

\begin{center}\rule{0.5\linewidth}{\linethickness}\end{center}

La \textbf{dimension 1} oppose des individus caractérisés par une
coordonnée fortement positive sur l'axe (à droite du graphe) à des
individus caractérisés par une coordonnée fortement négative sur l'axe
(à gauche du graphe).

Le groupe 1 (caractérisés par une coordonnée positive sur l'axe) partage
:

\begin{itemize}
\tightlist
\item
  de fortes valeurs pour des variables telles que \emph{Energy\_kcal},
  \emph{Phosphorus\_mg}, \emph{Protein\_gm}, \emph{Potassium\_mg},
  \emph{Total\_fat\_gm},
  \emph{Total\_monounsaturated\_fatty\_acids\_gm}, \emph{Sodium\_mg},
  \emph{Selenium\_mcg}, \emph{Total\_saturated\_fatty\_acids\_gm} et
  \emph{Carbohydrate\_gm} (de la plus extrême à la moins extrême).
\end{itemize}

Le groupe 2 (caractérisés par une coordonnée positive sur l'axe) partage
:

\begin{itemize}
\tightlist
\item
  de fortes valeurs pour des variables telles que
  \emph{Folate\_DFE\_mcg}, \emph{Total\_folate\_mcg},
  \emph{Folic\_acid\_mcg}, \emph{Iron\_mg}, \emph{Vitamin\_A\_RAE\_mcg},
  \emph{Vitamin\_B12\_mcg}, \emph{Retinol\_mcg},
  \emph{Riboflavin\_Vitamin\_B2\_mg}, \emph{Vitamin\_B6\_mg} et
  \emph{Zinc\_mg} (de la plus extrême à la moins extrême).
\end{itemize}

Le groupe 3 (caractérisés par une coordonnées négative sur l'axe)
partage :

\begin{itemize}
\tightlist
\item
  de faibles valeurs pour des variables telles que
  \emph{Phosphorus\_mg}, \emph{Energy\_kcal}, \emph{Potassium\_mg},
  \emph{Protein\_gm}, \emph{Magnesium\_mg},
  \emph{Thiamin\_Vitamin\_B1\_mg}, \emph{Riboflavin\_Vitamin\_B2\_mg},
  \emph{Zinc\_mg}, \emph{Niacin\_mg} et \emph{Selenium\_mcg} (de la plus
  extrême à la moins extrême).
\end{itemize}

\begin{center}\rule{0.5\linewidth}{\linethickness}\end{center}

La \textbf{dimension 2} oppose des individus caractérisés par une
coordonnée fortement positive sur l'axe (en haut du graphe) à des
individus caractérisés par une coordonnée fortement négative sur l'axe
(en bas du graphe).

Le groupe 1 (caractérisés par une coordonnée positive sur l'axe) partage
:

\begin{itemize}
\tightlist
\item
  de faibles valeurs pour des variables telles que
  \emph{Phosphorus\_mg}, \emph{Energy\_kcal}, \emph{Potassium\_mg},
  \emph{Protein\_gm}, \emph{Magnesium\_mg},
  \emph{Thiamin\_Vitamin\_B1\_mg}, \emph{Riboflavin\_Vitamin\_B2\_mg},
  \emph{Zinc\_mg}, \emph{Niacin\_mg} et \emph{Selenium\_mcg} (de la plus
  extrême à la moins extrême).
\end{itemize}

Le groupe 2 (caractérisés par une coordonnée positive sur l'axe) partage
:

\begin{itemize}
\tightlist
\item
  de fortes valeurs pour des variables telles que
  \emph{Folate\_DFE\_mcg}, \emph{Total\_folate\_mcg},
  \emph{Folic\_acid\_mcg}, \emph{Iron\_mg}, \emph{Vitamin\_A\_RAE\_mcg},
  \emph{Vitamin\_B12\_mcg}, \emph{Retinol\_mcg},
  \emph{Riboflavin\_Vitamin\_B2\_mg}, \emph{Vitamin\_B6\_mg} et
  \emph{Zinc\_mg} (de la plus extrême à la moins extrême).
\end{itemize}

Le groupe 3 (caractérisés par une coordonnées négative sur l'axe)
partage :

\begin{itemize}
\tightlist
\item
  de fortes valeurs pour des variables telles que \emph{Energy\_kcal},
  \emph{Phosphorus\_mg}, \emph{Protein\_gm}, \emph{Potassium\_mg},
  \emph{Total\_fat\_gm},
  \emph{Total\_monounsaturated\_fatty\_acids\_gm}, \emph{Sodium\_mg},
  \emph{Selenium\_mcg}, \emph{Total\_saturated\_fatty\_acids\_gm} et
  \emph{Carbohydrate\_gm} (de la plus extrême à la moins extrême).
\end{itemize}

\begin{center}\rule{0.5\linewidth}{\linethickness}\end{center}

\subsubsection{4. Description du plan 3:4}\label{description-du-plan-34}

\begin{center}\includegraphics{PCA_files/figure-latex/unnamed-chunk-5-1} \end{center}

\textbf{Figure 4.1 - Graphe des individus (ACP)} \emph{Les individus
libellés sont ceux ayant la plus grande contribution à la construction
du plan.}

\begin{center}\includegraphics{PCA_files/figure-latex/unnamed-chunk-6-1} \end{center}

\textbf{Figure 4.2 - Graphe des variables (ACP)} \emph{Les variables
libellées sont celles les mieux représentées sur le plan.}

\begin{center}\rule{0.5\linewidth}{\linethickness}\end{center}

La \textbf{dimension 3} oppose des individus caractérisés par une
coordonnée fortement positive sur l'axe (à droite du graphe) à des
individus caractérisés par une coordonnée fortement négative sur l'axe
(à gauche du graphe).

Le groupe 1 (caractérisés par une coordonnée positive sur l'axe) partage
:

\begin{itemize}
\tightlist
\item
  de fortes valeurs pour des variables telles que
  \emph{beta\_carotene\_mcg}, \emph{Vitamin\_K\_mcg},
  \emph{Lutein\_zeaxanthin\_mcg}, \emph{Food\_folate\_mcg},
  \emph{Vitamin\_C\_mg}, \emph{alpha\_carotene\_mcg},
  \emph{Potassium\_mg}, \emph{Dietary\_fiber\_gm}, \emph{Magnesium\_mg}
  et \emph{Moisture\_gm} (de la plus extrême à la moins extrême).
\item
  de faibles valeurs pour des variables telles que
  \emph{Folic\_acid\_mcg}, \emph{Added\_vitamin\_B12\_mcg},
  \emph{Thiamin\_Vitamin\_B1\_mg}, \emph{Folate\_DFE\_mcg},
  \emph{Vitamin\_B12\_mcg}, \emph{Iron\_mg}, \emph{Niacin\_mg},
  \emph{Riboflavin\_Vitamin\_B2\_mg}, \emph{Retinol\_mcg} et
  \emph{Vitamin\_D\_D2\_D3\_mcg} (de la plus extrême à la moins
  extrême).
\end{itemize}

Le groupe 2 (caractérisés par une coordonnées négative sur l'axe)
partage :

\begin{itemize}
\tightlist
\item
  de fortes valeurs pour des variables telles que
  \emph{Folic\_acid\_mcg}, \emph{Folate\_DFE\_mcg},
  \emph{Carbohydrate\_gm}, \emph{Thiamin\_Vitamin\_B1\_mg},
  \emph{Iron\_mg}, \emph{Total\_folate\_mcg},
  \emph{Added\_vitamin\_B12\_mcg}, \emph{Total\_sugars\_gm},
  \emph{Niacin\_mg} et \emph{Vitamin\_B6\_mg} (de la plus extrême à la
  moins extrême).
\item
  de faibles valeurs pour des variables telles que
  \emph{Total\_choline\_mg}, \emph{Cholesterol\_mg},
  \emph{beta\_carotene\_mcg}, \emph{Vitamin\_K\_mcg},
  \emph{Lutein\_zeaxanthin\_mcg}, \emph{Vitamin\_A\_RAE\_mcg},
  \emph{Protein\_gm}, \emph{alpha\_carotene\_mcg}, \emph{Potassium\_mg}
  et \emph{Moisture\_gm} (de la plus extrême à la moins extrême).
\end{itemize}

Le groupe 3 (caractérisés par une coordonnées négative sur l'axe)
partage :

\begin{itemize}
\tightlist
\item
  de fortes valeurs pour des variables telles que
  \emph{Cholesterol\_mg}, \emph{Total\_choline\_mg},
  \emph{Vitamin\_D\_D2\_D3\_mcg}, \emph{Protein\_gm},
  \emph{Selenium\_mcg}, \emph{Retinol\_mcg}, \emph{Vitamin\_B12\_mcg},
  \emph{Phosphorus\_mg}, \emph{Total\_saturated\_fatty\_acids\_gm} et
  \emph{Total\_monounsaturated\_fatty\_acids\_gm} (de la plus extrême à
  la moins extrême).
\item
  de faibles valeurs pour des variables telles que
  \emph{Dietary\_fiber\_gm}, \emph{Carbohydrate\_gm},
  \emph{Total\_folate\_mcg}, \emph{Folate\_DFE\_mcg},
  \emph{Vitamin\_C\_mg}, \emph{Folic\_acid\_mcg}, \emph{Iron\_mg},
  \emph{Thiamin\_Vitamin\_B1\_mg}, \emph{Food\_folate\_mcg} et
  \emph{Total\_sugars\_gm} (de la plus extrême à la moins extrême).
\end{itemize}

\begin{center}\rule{0.5\linewidth}{\linethickness}\end{center}

La \textbf{dimension 4} oppose des individus caractérisés par une
coordonnée fortement positive sur l'axe (en haut du graphe) à des
individus caractérisés par une coordonnée fortement négative sur l'axe
(en bas du graphe).

Le groupe 1 (caractérisés par une coordonnée positive sur l'axe) partage
:

\begin{itemize}
\tightlist
\item
  de fortes valeurs pour des variables telles que
  \emph{Cholesterol\_mg}, \emph{Total\_choline\_mg},
  \emph{Vitamin\_D\_D2\_D3\_mcg}, \emph{Protein\_gm},
  \emph{Selenium\_mcg}, \emph{Retinol\_mcg}, \emph{Vitamin\_B12\_mcg},
  \emph{Phosphorus\_mg}, \emph{Total\_saturated\_fatty\_acids\_gm} et
  \emph{Total\_monounsaturated\_fatty\_acids\_gm} (de la plus extrême à
  la moins extrême).
\item
  de faibles valeurs pour des variables telles que
  \emph{Dietary\_fiber\_gm}, \emph{Carbohydrate\_gm},
  \emph{Total\_folate\_mcg}, \emph{Folate\_DFE\_mcg},
  \emph{Vitamin\_C\_mg}, \emph{Folic\_acid\_mcg}, \emph{Iron\_mg},
  \emph{Thiamin\_Vitamin\_B1\_mg}, \emph{Food\_folate\_mcg} et
  \emph{Total\_sugars\_gm} (de la plus extrême à la moins extrême).
\end{itemize}

Le groupe 2 (caractérisés par une coordonnée positive sur l'axe) partage
:

\begin{itemize}
\tightlist
\item
  de fortes valeurs pour des variables telles que
  \emph{beta\_carotene\_mcg}, \emph{Vitamin\_K\_mcg},
  \emph{Lutein\_zeaxanthin\_mcg}, \emph{Food\_folate\_mcg},
  \emph{Vitamin\_C\_mg}, \emph{alpha\_carotene\_mcg},
  \emph{Potassium\_mg}, \emph{Dietary\_fiber\_gm}, \emph{Magnesium\_mg}
  et \emph{Moisture\_gm} (de la plus extrême à la moins extrême).
\item
  de faibles valeurs pour des variables telles que
  \emph{Folic\_acid\_mcg}, \emph{Added\_vitamin\_B12\_mcg},
  \emph{Thiamin\_Vitamin\_B1\_mg}, \emph{Folate\_DFE\_mcg},
  \emph{Vitamin\_B12\_mcg}, \emph{Iron\_mg}, \emph{Niacin\_mg},
  \emph{Riboflavin\_Vitamin\_B2\_mg}, \emph{Retinol\_mcg} et
  \emph{Vitamin\_D\_D2\_D3\_mcg} (de la plus extrême à la moins
  extrême).
\end{itemize}

Le groupe 3 (caractérisés par une coordonnées négative sur l'axe)
partage :

\begin{itemize}
\tightlist
\item
  de fortes valeurs pour des variables telles que
  \emph{Folic\_acid\_mcg}, \emph{Folate\_DFE\_mcg},
  \emph{Carbohydrate\_gm}, \emph{Thiamin\_Vitamin\_B1\_mg},
  \emph{Iron\_mg}, \emph{Total\_folate\_mcg},
  \emph{Added\_vitamin\_B12\_mcg}, \emph{Total\_sugars\_gm},
  \emph{Niacin\_mg} et \emph{Vitamin\_B6\_mg} (de la plus extrême à la
  moins extrême).
\item
  de faibles valeurs pour des variables telles que
  \emph{Total\_choline\_mg}, \emph{Cholesterol\_mg},
  \emph{beta\_carotene\_mcg}, \emph{Vitamin\_K\_mcg},
  \emph{Lutein\_zeaxanthin\_mcg}, \emph{Vitamin\_A\_RAE\_mcg},
  \emph{Protein\_gm}, \emph{alpha\_carotene\_mcg}, \emph{Potassium\_mg}
  et \emph{Moisture\_gm} (de la plus extrême à la moins extrême).
\end{itemize}

\begin{center}\rule{0.5\linewidth}{\linethickness}\end{center}

\subsubsection{5. Description du plan 5:6}\label{description-du-plan-56}

\begin{center}\includegraphics{PCA_files/figure-latex/unnamed-chunk-7-1} \end{center}

\textbf{Figure 5.1 - Graphe des individus (ACP)} \emph{Les individus
libellés sont ceux ayant la plus grande contribution à la construction
du plan.}

\begin{center}\includegraphics{PCA_files/figure-latex/unnamed-chunk-8-1} \end{center}

\textbf{Figure 5.2 - Graphe des variables (ACP)} \emph{Les variables
libellées sont celles les mieux représentées sur le plan.}

\begin{center}\rule{0.5\linewidth}{\linethickness}\end{center}

La \textbf{dimension 5} oppose des individus caractérisés par une
coordonnée fortement positive sur l'axe (à droite du graphe) à des
individus caractérisés par une coordonnée fortement négative sur l'axe
(à gauche du graphe).

Le groupe 1 (caractérisés par une coordonnée positive sur l'axe) partage
:

\begin{itemize}
\tightlist
\item
  de fortes valeurs pour des variables telles que
  \emph{Total\_sugars\_gm}, \emph{Theobromine\_mg},
  \emph{Carbohydrate\_gm}, \emph{Total\_saturated\_fatty\_acids\_gm},
  \emph{Vitamin\_A\_RAE\_mcg}, \emph{Retinol\_mcg}, \emph{Caffeine\_mg},
  \emph{Total\_fat\_gm}, \emph{Energy\_kcal} et \emph{Calcium\_mg} (de
  la plus extrême à la moins extrême).
\item
  de faibles valeurs pour des variables telles que \emph{Selenium\_mcg},
  \emph{Protein\_gm}, \emph{Total\_choline\_mg}, \emph{Niacin\_mg},
  \emph{Cholesterol\_mg}, \emph{Alcohol\_gm}, \emph{Total\_folate\_mcg},
  \emph{Vitamin\_B6\_mg}, \emph{Sodium\_mg} et \emph{Folate\_DFE\_mcg}
  (de la plus extrême à la moins extrême).
\end{itemize}

Le groupe 2 (caractérisés par une coordonnées négative sur l'axe)
partage :

\begin{itemize}
\tightlist
\item
  de fortes valeurs pour des variables telles que \emph{Alcohol\_gm},
  \emph{Moisture\_gm}, \emph{Vitamin\_B6\_mg}, \emph{Niacin\_mg},
  \emph{Added\_vitamin\_B12\_mcg}, \emph{Food\_folate\_mcg},
  \emph{Riboflavin\_Vitamin\_B2\_mg}, \emph{Vitamin\_B12\_mcg},
  \emph{Magnesium\_mg} et \emph{Total\_choline\_mg} (de la plus extrême
  à la moins extrême).
\item
  de faibles valeurs pour des variables telles que \emph{Retinol\_mcg},
  \emph{Total\_saturated\_fatty\_acids\_gm}, \emph{Folic\_acid\_mcg},
  \emph{Total\_fat\_gm}, \emph{Vitamin\_A\_RAE\_mcg},
  \emph{Theobromine\_mg},
  \emph{Total\_monounsaturated\_fatty\_acids\_gm}, \emph{Iron\_mg},
  \emph{Total\_polyunsaturated\_fatty\_acids\_gm} et
  \emph{Total\_sugars\_gm} (de la plus extrême à la moins extrême).
\end{itemize}

Le groupe 3 (caractérisés par une coordonnées négative sur l'axe)
partage :

\begin{itemize}
\tightlist
\item
  de fortes valeurs pour des variables telles que \emph{Selenium\_mcg},
  \emph{Folic\_acid\_mcg}, \emph{Cholesterol\_mg},
  \emph{Folate\_DFE\_mcg}, \emph{Protein\_gm},
  \emph{Total\_folate\_mcg}, \emph{Sodium\_mg},
  \emph{Thiamin\_Vitamin\_B1\_mg}, \emph{Iron\_mg} et \emph{Zinc\_mg}
  (de la plus extrême à la moins extrême).
\item
  de faibles valeurs pour des variables telles que
  \emph{Total\_sugars\_gm}, \emph{Caffeine\_mg}, \emph{Moisture\_gm},
  \emph{Theobromine\_mg}, \emph{Carbohydrate\_gm},
  \emph{Riboflavin\_Vitamin\_B2\_mg}, \emph{Potassium\_mg},
  \emph{Energy\_kcal}, \emph{Alcohol\_gm} et \emph{Magnesium\_mg} (de la
  plus extrême à la moins extrême).
\end{itemize}

\begin{center}\rule{0.5\linewidth}{\linethickness}\end{center}

La \textbf{dimension 6} oppose des individus caractérisés par une
coordonnée fortement positive sur l'axe (en haut du graphe) à des
individus caractérisés par une coordonnée fortement négative sur l'axe
(en bas du graphe).

Le groupe 1 (caractérisés par une coordonnée positive sur l'axe) partage
:

\begin{itemize}
\tightlist
\item
  de fortes valeurs pour des variables telles que \emph{Alcohol\_gm},
  \emph{Moisture\_gm}, \emph{Vitamin\_B6\_mg}, \emph{Niacin\_mg},
  \emph{Added\_vitamin\_B12\_mcg}, \emph{Food\_folate\_mcg},
  \emph{Riboflavin\_Vitamin\_B2\_mg}, \emph{Vitamin\_B12\_mcg},
  \emph{Magnesium\_mg} et \emph{Total\_choline\_mg} (de la plus extrême
  à la moins extrême).
\item
  de faibles valeurs pour des variables telles que \emph{Retinol\_mcg},
  \emph{Total\_saturated\_fatty\_acids\_gm}, \emph{Folic\_acid\_mcg},
  \emph{Total\_fat\_gm}, \emph{Vitamin\_A\_RAE\_mcg},
  \emph{Theobromine\_mg},
  \emph{Total\_monounsaturated\_fatty\_acids\_gm}, \emph{Iron\_mg},
  \emph{Total\_polyunsaturated\_fatty\_acids\_gm} et
  \emph{Total\_sugars\_gm} (de la plus extrême à la moins extrême).
\end{itemize}

Le groupe 2 (caractérisés par une coordonnée positive sur l'axe) partage
:

\begin{itemize}
\tightlist
\item
  de fortes valeurs pour des variables telles que
  \emph{Total\_sugars\_gm}, \emph{Theobromine\_mg},
  \emph{Carbohydrate\_gm}, \emph{Total\_saturated\_fatty\_acids\_gm},
  \emph{Vitamin\_A\_RAE\_mcg}, \emph{Retinol\_mcg}, \emph{Caffeine\_mg},
  \emph{Total\_fat\_gm}, \emph{Energy\_kcal} et \emph{Calcium\_mg} (de
  la plus extrême à la moins extrême).
\item
  de faibles valeurs pour des variables telles que \emph{Selenium\_mcg},
  \emph{Protein\_gm}, \emph{Total\_choline\_mg}, \emph{Niacin\_mg},
  \emph{Cholesterol\_mg}, \emph{Alcohol\_gm}, \emph{Total\_folate\_mcg},
  \emph{Vitamin\_B6\_mg}, \emph{Sodium\_mg} et \emph{Folate\_DFE\_mcg}
  (de la plus extrême à la moins extrême).
\end{itemize}

Le groupe 3 (caractérisés par une coordonnées négative sur l'axe)
partage :

\begin{itemize}
\tightlist
\item
  de fortes valeurs pour des variables telles que \emph{Selenium\_mcg},
  \emph{Folic\_acid\_mcg}, \emph{Cholesterol\_mg},
  \emph{Folate\_DFE\_mcg}, \emph{Protein\_gm},
  \emph{Total\_folate\_mcg}, \emph{Sodium\_mg},
  \emph{Thiamin\_Vitamin\_B1\_mg}, \emph{Iron\_mg} et \emph{Zinc\_mg}
  (de la plus extrême à la moins extrême).
\item
  de faibles valeurs pour des variables telles que
  \emph{Total\_sugars\_gm}, \emph{Caffeine\_mg}, \emph{Moisture\_gm},
  \emph{Theobromine\_mg}, \emph{Carbohydrate\_gm},
  \emph{Riboflavin\_Vitamin\_B2\_mg}, \emph{Potassium\_mg},
  \emph{Energy\_kcal}, \emph{Alcohol\_gm} et \emph{Magnesium\_mg} (de la
  plus extrême à la moins extrême).
\end{itemize}

\begin{center}\rule{0.5\linewidth}{\linethickness}\end{center}

\subsubsection{6. Description du plan 7:8}\label{description-du-plan-78}

\begin{center}\includegraphics{PCA_files/figure-latex/unnamed-chunk-9-1} \end{center}

\textbf{Figure 6.1 - Graphe des individus (ACP)} \emph{Les individus
libellés sont ceux ayant la plus grande contribution à la construction
du plan.}

\begin{center}\includegraphics{PCA_files/figure-latex/unnamed-chunk-10-1} \end{center}

\textbf{Figure 6.2 - Graphe des variables (ACP)} \emph{Les variables
libellées sont celles les mieux représentées sur le plan.}

\begin{center}\rule{0.5\linewidth}{\linethickness}\end{center}

La \textbf{dimension 7} oppose des individus caractérisés par une
coordonnée fortement positive sur l'axe (à droite du graphe) à des
individus caractérisés par une coordonnée fortement négative sur l'axe
(à gauche du graphe).

Le groupe 1 (caractérisés par une coordonnée positive sur l'axe) partage
:

\begin{itemize}
\tightlist
\item
  de fortes valeurs pour des variables telles que \emph{tocopherol\_mg},
  \emph{Total\_polyunsaturated\_fatty\_acids\_gm},
  \emph{Total\_monounsaturated\_fatty\_acids\_gm},
  \emph{tocopherol\_Vitamin\_E\_mg}, \emph{Total\_fat\_gm},
  \emph{Vitamin\_K\_mcg}, \emph{Added\_vitamin\_B12\_mcg},
  \emph{Lutein\_zeaxanthin\_mcg}, \emph{Niacin\_mg} et
  \emph{Vitamin\_B6\_mg} (de la plus extrême à la moins extrême).
\item
  de faibles valeurs pour des variables telles que
  \emph{cryptoxanthin\_mcg}, \emph{alpha\_carotene\_mcg},
  \emph{Lycopene\_mcg}, \emph{Food\_folate\_mcg},
  \emph{Dietary\_fiber\_gm}, \emph{Calcium\_mg}, \emph{Potassium\_mg},
  \emph{Vitamin\_C\_mg}, \emph{Total\_folate\_mcg} et
  \emph{Thiamin\_Vitamin\_B1\_mg} (de la plus extrême à la moins
  extrême).
\end{itemize}

Le groupe 2 (caractérisés par une coordonnée positive sur l'axe) partage
:

\begin{itemize}
\tightlist
\item
  de fortes valeurs pour des variables telles que
  \emph{alpha\_carotene\_mcg}, \emph{beta\_carotene\_mcg},
  \emph{Folic\_acid\_mcg}, \emph{Sodium\_mg},
  \emph{Vitamin\_A\_RAE\_mcg}, \emph{Thiamin\_Vitamin\_B1\_mg},
  \emph{Folate\_DFE\_mcg}, \emph{Iron\_mg}, \emph{Selenium\_mcg} et
  \emph{Total\_folate\_mcg} (de la plus extrême à la moins extrême).
\item
  de faibles valeurs pour des variables telles que \emph{Magnesium\_mg},
  \emph{Vitamin\_C\_mg}, \emph{tocopherol\_mg},
  \emph{tocopherol\_Vitamin\_E\_mg}, \emph{Calcium\_mg},
  \emph{Food\_folate\_mcg}, \emph{Vitamin\_D\_D2\_D3\_mcg},
  \emph{Moisture\_gm}, \emph{Copper\_mg} et \emph{cryptoxanthin\_mcg}
  (de la plus extrême à la moins extrême).
\end{itemize}

Le groupe 3 (caractérisés par une coordonnées négative sur l'axe)
partage :

\begin{itemize}
\tightlist
\item
  de fortes valeurs pour des variables telles que \emph{Vitamin\_C\_mg},
  \emph{Calcium\_mg}, \emph{cryptoxanthin\_mcg},
  \emph{Food\_folate\_mcg}, \emph{Dietary\_fiber\_gm},
  \emph{Lycopene\_mcg}, \emph{Potassium\_mg},
  \emph{Vitamin\_D\_D2\_D3\_mcg}, \emph{Moisture\_gm} et
  \emph{Magnesium\_mg} (de la plus extrême à la moins extrême).
\item
  de faibles valeurs pour des variables telles que
  \emph{Total\_polyunsaturated\_fatty\_acids\_gm},
  \emph{Total\_monounsaturated\_fatty\_acids\_gm},
  \emph{Total\_fat\_gm}, \emph{tocopherol\_mg}, \emph{Niacin\_mg},
  \emph{Added\_vitamin\_B12\_mcg}, \emph{Vitamin\_K\_mcg},
  \emph{beta\_carotene\_mcg}, \emph{Sodium\_mg} et
  \emph{Vitamin\_B6\_mg} (de la plus extrême à la moins extrême).
\end{itemize}

\begin{center}\rule{0.5\linewidth}{\linethickness}\end{center}

La \textbf{dimension 8} oppose des individus caractérisés par une
coordonnée fortement positive sur l'axe (en haut du graphe) à des
individus caractérisés par une coordonnée fortement négative sur l'axe
(en bas du graphe).

Le groupe 1 (caractérisés par une coordonnée positive sur l'axe) partage
:

\begin{itemize}
\tightlist
\item
  de fortes valeurs pour des variables telles que \emph{tocopherol\_mg},
  \emph{Total\_polyunsaturated\_fatty\_acids\_gm},
  \emph{Total\_monounsaturated\_fatty\_acids\_gm},
  \emph{tocopherol\_Vitamin\_E\_mg}, \emph{Total\_fat\_gm},
  \emph{Vitamin\_K\_mcg}, \emph{Added\_vitamin\_B12\_mcg},
  \emph{Lutein\_zeaxanthin\_mcg}, \emph{Niacin\_mg} et
  \emph{Vitamin\_B6\_mg} (de la plus extrême à la moins extrême).
\item
  de faibles valeurs pour des variables telles que
  \emph{cryptoxanthin\_mcg}, \emph{alpha\_carotene\_mcg},
  \emph{Lycopene\_mcg}, \emph{Food\_folate\_mcg},
  \emph{Dietary\_fiber\_gm}, \emph{Calcium\_mg}, \emph{Potassium\_mg},
  \emph{Vitamin\_C\_mg}, \emph{Total\_folate\_mcg} et
  \emph{Thiamin\_Vitamin\_B1\_mg} (de la plus extrême à la moins
  extrême).
\end{itemize}

Le groupe 2 (caractérisés par une coordonnées négative sur l'axe)
partage :

\begin{itemize}
\tightlist
\item
  de fortes valeurs pour des variables telles que
  \emph{alpha\_carotene\_mcg}, \emph{beta\_carotene\_mcg},
  \emph{Folic\_acid\_mcg}, \emph{Sodium\_mg},
  \emph{Vitamin\_A\_RAE\_mcg}, \emph{Thiamin\_Vitamin\_B1\_mg},
  \emph{Folate\_DFE\_mcg}, \emph{Iron\_mg}, \emph{Selenium\_mcg} et
  \emph{Total\_folate\_mcg} (de la plus extrême à la moins extrême).
\item
  de faibles valeurs pour des variables telles que \emph{Magnesium\_mg},
  \emph{Vitamin\_C\_mg}, \emph{tocopherol\_mg},
  \emph{tocopherol\_Vitamin\_E\_mg}, \emph{Calcium\_mg},
  \emph{Food\_folate\_mcg}, \emph{Vitamin\_D\_D2\_D3\_mcg},
  \emph{Moisture\_gm}, \emph{Copper\_mg} et \emph{cryptoxanthin\_mcg}
  (de la plus extrême à la moins extrême).
\end{itemize}

\begin{center}\rule{0.5\linewidth}{\linethickness}\end{center}

\subsubsection{7. Description du plan
9:10}\label{description-du-plan-910}

\begin{center}\includegraphics{PCA_files/figure-latex/unnamed-chunk-11-1} \end{center}

\textbf{Figure 7.1 - Graphe des individus (ACP)} \emph{Les individus
libellés sont ceux ayant la plus grande contribution à la construction
du plan.}

\begin{center}\includegraphics{PCA_files/figure-latex/unnamed-chunk-12-1} \end{center}

\textbf{Figure 7.2 - Graphe des variables (ACP)} \emph{Les variables
libellées sont celles les mieux représentées sur le plan.}

\begin{center}\rule{0.5\linewidth}{\linethickness}\end{center}

La \textbf{dimension 9} oppose des individus caractérisés par une
coordonnée fortement positive sur l'axe (à droite du graphe) à des
individus caractérisés par une coordonnée fortement négative sur l'axe
(à gauche du graphe).

Le groupe 1 (caractérisés par une coordonnée positive sur l'axe) partage
:

\begin{itemize}
\tightlist
\item
  de fortes valeurs pour des variables telles que \emph{Vitamin\_C\_mg},
  \emph{Total\_sugars\_gm}, \emph{cryptoxanthin\_mcg},
  \emph{Cholesterol\_mg}, \emph{Vitamin\_B6\_mg},
  \emph{Carbohydrate\_gm}, \emph{Vitamin\_D\_D2\_D3\_mcg},
  \emph{Vitamin\_B12\_mcg}, \emph{Added\_vitamin\_B12\_mcg} et
  \emph{Potassium\_mg} (de la plus extrême à la moins extrême).
\item
  de faibles valeurs pour des variables telles que \emph{Moisture\_gm},
  \emph{Caffeine\_mg}, \emph{Alcohol\_gm}, \emph{Folate\_DFE\_mcg},
  \emph{Total\_folate\_mcg}, \emph{Folic\_acid\_mcg},
  \emph{tocopherol\_mg}, \emph{Magnesium\_mg},
  \emph{tocopherol\_Vitamin\_E\_mg} et \emph{Copper\_mg} (de la plus
  extrême à la moins extrême).
\end{itemize}

Le groupe 2 (caractérisés par une coordonnées négative sur l'axe)
partage :

\begin{itemize}
\tightlist
\item
  de fortes valeurs pour des variables telles que \emph{tocopherol\_mg},
  \emph{tocopherol\_Vitamin\_E\_mg}, \emph{alpha\_carotene\_mcg},
  \emph{Vitamin\_A\_RAE\_mcg}, \emph{Copper\_mg},
  \emph{beta\_carotene\_mcg}, \emph{Magnesium\_mg},
  \emph{Dietary\_fiber\_gm},
  \emph{Total\_polyunsaturated\_fatty\_acids\_gm} et
  \emph{Total\_monounsaturated\_fatty\_acids\_gm} (de la plus extrême à
  la moins extrême).
\item
  de faibles valeurs pour les variables \emph{Total\_sugars\_gm},
  \emph{Cholesterol\_mg}, \emph{Lutein\_zeaxanthin\_mcg},
  \emph{Carbohydrate\_gm}, \emph{cryptoxanthin\_mcg},
  \emph{Caffeine\_mg}, \emph{Vitamin\_D\_D2\_D3\_mcg} et
  \emph{Vitamin\_K\_mcg} (de la plus extrême à la moins extrême).
\end{itemize}

Le groupe 3 (caractérisés par une coordonnées négative sur l'axe)
partage :

\begin{itemize}
\tightlist
\item
  de fortes valeurs pour des variables telles que \emph{Caffeine\_mg},
  \emph{Alcohol\_gm}, \emph{Moisture\_gm}, \emph{Total\_folate\_mcg},
  \emph{Folate\_DFE\_mcg}, \emph{Folic\_acid\_mcg},
  \emph{Lutein\_zeaxanthin\_mcg}, \emph{Vitamin\_K\_mcg},
  \emph{Food\_folate\_mcg} et \emph{Total\_saturated\_fatty\_acids\_gm}
  (de la plus extrême à la moins extrême).
\item
  de faibles valeurs pour des variables telles que
  \emph{Vitamin\_C\_mg}, \emph{Total\_sugars\_gm},
  \emph{cryptoxanthin\_mcg}, \emph{Vitamin\_B6\_mg},
  \emph{tocopherol\_mg}, \emph{tocopherol\_Vitamin\_E\_mg},
  \emph{Vitamin\_B12\_mcg}, \emph{alpha\_carotene\_mcg},
  \emph{Lycopene\_mcg} et \emph{Added\_vitamin\_B12\_mcg} (de la plus
  extrême à la moins extrême).
\end{itemize}

\begin{center}\rule{0.5\linewidth}{\linethickness}\end{center}

La \textbf{dimension 10} oppose des individus caractérisés par une
coordonnée fortement positive sur l'axe (en haut du graphe) à des
individus caractérisés par une coordonnée fortement négative sur l'axe
(en bas du graphe).

Le groupe 1 (caractérisés par une coordonnée positive sur l'axe) partage
:

\begin{itemize}
\tightlist
\item
  de fortes valeurs pour des variables telles que \emph{tocopherol\_mg},
  \emph{tocopherol\_Vitamin\_E\_mg}, \emph{alpha\_carotene\_mcg},
  \emph{Vitamin\_A\_RAE\_mcg}, \emph{Copper\_mg},
  \emph{beta\_carotene\_mcg}, \emph{Magnesium\_mg},
  \emph{Dietary\_fiber\_gm},
  \emph{Total\_polyunsaturated\_fatty\_acids\_gm} et
  \emph{Total\_monounsaturated\_fatty\_acids\_gm} (de la plus extrême à
  la moins extrême).
\item
  de faibles valeurs pour les variables \emph{Total\_sugars\_gm},
  \emph{Cholesterol\_mg}, \emph{Lutein\_zeaxanthin\_mcg},
  \emph{Carbohydrate\_gm}, \emph{cryptoxanthin\_mcg},
  \emph{Caffeine\_mg}, \emph{Vitamin\_D\_D2\_D3\_mcg} et
  \emph{Vitamin\_K\_mcg} (de la plus extrême à la moins extrême).
\end{itemize}

Le groupe 2 (caractérisés par une coordonnée positive sur l'axe) partage
:

\begin{itemize}
\tightlist
\item
  de fortes valeurs pour des variables telles que \emph{Vitamin\_C\_mg},
  \emph{Total\_sugars\_gm}, \emph{cryptoxanthin\_mcg},
  \emph{Cholesterol\_mg}, \emph{Vitamin\_B6\_mg},
  \emph{Carbohydrate\_gm}, \emph{Vitamin\_D\_D2\_D3\_mcg},
  \emph{Vitamin\_B12\_mcg}, \emph{Added\_vitamin\_B12\_mcg} et
  \emph{Potassium\_mg} (de la plus extrême à la moins extrême).
\item
  de faibles valeurs pour des variables telles que \emph{Moisture\_gm},
  \emph{Caffeine\_mg}, \emph{Alcohol\_gm}, \emph{Folate\_DFE\_mcg},
  \emph{Total\_folate\_mcg}, \emph{Folic\_acid\_mcg},
  \emph{tocopherol\_mg}, \emph{Magnesium\_mg},
  \emph{tocopherol\_Vitamin\_E\_mg} et \emph{Copper\_mg} (de la plus
  extrême à la moins extrême).
\end{itemize}

Le groupe 3 (caractérisés par une coordonnées négative sur l'axe)
partage :

\begin{itemize}
\tightlist
\item
  de fortes valeurs pour des variables telles que \emph{Caffeine\_mg},
  \emph{Alcohol\_gm}, \emph{Moisture\_gm}, \emph{Total\_folate\_mcg},
  \emph{Folate\_DFE\_mcg}, \emph{Folic\_acid\_mcg},
  \emph{Lutein\_zeaxanthin\_mcg}, \emph{Vitamin\_K\_mcg},
  \emph{Food\_folate\_mcg} et \emph{Total\_saturated\_fatty\_acids\_gm}
  (de la plus extrême à la moins extrême).
\item
  de faibles valeurs pour des variables telles que
  \emph{Vitamin\_C\_mg}, \emph{Total\_sugars\_gm},
  \emph{cryptoxanthin\_mcg}, \emph{Vitamin\_B6\_mg},
  \emph{tocopherol\_mg}, \emph{tocopherol\_Vitamin\_E\_mg},
  \emph{Vitamin\_B12\_mcg}, \emph{alpha\_carotene\_mcg},
  \emph{Lycopene\_mcg} et \emph{Added\_vitamin\_B12\_mcg} (de la plus
  extrême à la moins extrême).
\end{itemize}

\begin{center}\rule{0.5\linewidth}{\linethickness}\end{center}

\subsubsection{8. Classification}\label{classification}

Le jeu de données est trop volumineux pour réaliser la classification.

\begin{center}\rule{0.5\linewidth}{\linethickness}\end{center}

\subsection{Annexes}\label{annexes}


\end{document}
